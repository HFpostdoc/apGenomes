% latex table generated in R 3.4.2 by xtable 1.8-2 package
% Sat Jan 27 13:32:26 2018
\begin{table}[ht]
\centering
\begin{tabular}{rrr}
  \hline
 & r & p \\ 
  \hline
MDR: Mean Diurnal Range (Mean of monthly (max temp - min temp)) (BIO2) & 0.270 & 0.540 \\ 
  PCQ: Precipitation of Coldest Quarter (BIO19) & 0.252 & 0.542 \\ 
  PA: Annual Precipitation (BIO12) & 0.203 & 0.659 \\ 
  PWaQ: Precipitation of Warmest Quarter (BIO18) & 0.173 & 0.703 \\ 
  PWeQ: Precipitation of Wettest Quarter (BIO16) & 0.173 & 0.707 \\ 
  PWM: Precipitation of Wettest Month (BIO13) & 0.166 & 0.721 \\ 
  ATR: Temperature Annual Range (BIO5-BIO6) (BIO7) & 0.128 & 0.771 \\ 
  MTWeQ: Mean Temperature of Wettest Quarter (BIO8) & 0.093 & 0.799 \\ 
  TS: Temperature Seasonality (standard deviation *100) (BIO4) & 0.110 & 0.801 \\ 
  Tmin: Min Temperature of Coldest Month (BIO6) & 0.100 & 0.802 \\ 
  PS: Precipitation Seasonality (Coefficient of Variation) (BIO15) & 0.127 & 0.802 \\ 
  Iso: Isothermality (BIO2/BIO7) (* 100) (BIO3) & 0.102 & 0.807 \\ 
  MTCQ: Mean Temperature of Coldest Quarter (BIO11) & 0.091 & 0.808 \\ 
  Latitude & 0.089 & 0.815 \\ 
  MAT: Annual Mean Temperature (BIO1) & 0.084 & 0.816 \\ 
  MTWaQ: Mean Temperature of Warmest Quarter (BIO10) & 0.064 & 0.845 \\ 
  Longitude & 0.046 & 0.902 \\ 
  Tmax: Max Temperature of Warmest Month (BIO5) & 0.039 & 0.912 \\ 
  MTDQ: Mean Temperature of Driest Quarter (BIO9) & 0.040 & 0.919 \\ 
  PDQ: Precipitation of Driest Quarter (BIO17) & 0.029 & 0.956 \\ 
  PDM: Precipitation of Driest Month (BIO14) & 0.024 & 0.959 \\ 
   \hline
\end{tabular}
\caption{Results of the NMS ordination vector analysis for only Aphaenogaster spp.} 
\label{tab:wc_apg_vec}
\end{table}
