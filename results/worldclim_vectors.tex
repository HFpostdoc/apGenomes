% latex table generated in R 3.4.2 by xtable 1.8-2 package
% Sat Jan 27 13:32:25 2018
\begin{table}[ht]
\centering
\begin{tabular}{rrr}
  \hline
 & r & p \\ 
  \hline
MDR: Mean Diurnal Range (Mean of monthly (max temp - min temp)) (BIO2) & 0.415 & 0.004 \\ 
  PCQ: Precipitation of Coldest Quarter (BIO19) & 0.360 & 0.006 \\ 
  ATR: Temperature Annual Range (BIO5-BIO6) (BIO7) & 0.349 & 0.010 \\ 
  PDM: Precipitation of Driest Month (BIO14) & 0.321 & 0.017 \\ 
  Iso: Isothermality (BIO2/BIO7) (* 100) (BIO3) & 0.296 & 0.025 \\ 
  Tmin: Min Temperature of Coldest Month (BIO6) & 0.269 & 0.037 \\ 
  PWM: Precipitation of Wettest Month (BIO13) & 0.259 & 0.039 \\ 
  PWeQ: Precipitation of Wettest Quarter (BIO16) & 0.240 & 0.054 \\ 
  PDQ: Precipitation of Driest Quarter (BIO17) & 0.242 & 0.055 \\ 
  MTDQ: Mean Temperature of Driest Quarter (BIO9) & 0.231 & 0.057 \\ 
  TS: Temperature Seasonality (standard deviation *100) (BIO4) & 0.231 & 0.060 \\ 
  Tmax: Max Temperature of Warmest Month (BIO5) & 0.219 & 0.066 \\ 
  Longitude & 0.210 & 0.071 \\ 
  MTCQ: Mean Temperature of Coldest Quarter (BIO11) & 0.205 & 0.088 \\ 
  PA: Annual Precipitation (BIO12) & 0.186 & 0.117 \\ 
  MAT: Annual Mean Temperature (BIO1) & 0.171 & 0.133 \\ 
  PS: Precipitation Seasonality (Coefficient of Variation) (BIO15) & 0.157 & 0.160 \\ 
  MTWaQ: Mean Temperature of Warmest Quarter (BIO10) & 0.107 & 0.303 \\ 
  Latitude & 0.094 & 0.350 \\ 
  PWaQ: Precipitation of Warmest Quarter (BIO18) & 0.056 & 0.553 \\ 
  MTWeQ: Mean Temperature of Wettest Quarter (BIO8) & 0.002 & 0.976 \\ 
   \hline
\end{tabular}
\caption{Results of the NMS ordination vector analysis.} 
\label{tab:wc_vec}
\end{table}
