% latex table generated in R 3.4.2 by xtable 1.8-2 package
% Sun Feb 18 20:11:44 2018
\begin{table}[ht]
\centering
\begin{tabular}{rrr}
  \hline
 & r & p \\ 
  \hline
PCQ: Precipitation of Coldest Quarter (BIO19) & 0.400 & 0.008 \\ 
  MTWaQ: Mean Temperature of Warmest Quarter (BIO10) & 0.216 & 0.098 \\ 
  PWM: Precipitation of Wettest Month (BIO13) & 0.216 & 0.100 \\ 
  PWeQ: Precipitation of Wettest Quarter (BIO16) & 0.206 & 0.114 \\ 
  PA: Annual Precipitation (BIO12) & 0.203 & 0.119 \\ 
  Tmax: Max Temperature of Warmest Month (BIO5) & 0.188 & 0.142 \\ 
  Iso: Isothermality (BIO2/BIO7) (* 100) (BIO3) & 0.184 & 0.145 \\ 
  Longitude & 0.178 & 0.153 \\ 
  MAT: Annual Mean Temperature (BIO1) & 0.176 & 0.159 \\ 
  Tmin: Min Temperature of Coldest Month (BIO6) & 0.166 & 0.181 \\ 
  MTDQ: Mean Temperature of Driest Quarter (BIO9) & 0.162 & 0.186 \\ 
  MTCQ: Mean Temperature of Coldest Quarter (BIO11) & 0.148 & 0.215 \\ 
  ATR: Temperature Annual Range (BIO5-BIO6) (BIO7) & 0.136 & 0.249 \\ 
  MDR: Mean Diurnal Range (Mean of monthly (max temp - min temp)) (BIO2) & 0.114 & 0.313 \\ 
  TS: Temperature Seasonality (standard deviation *100) (BIO4) & 0.089 & 0.412 \\ 
  PDM: Precipitation of Driest Month (BIO14) & 0.090 & 0.413 \\ 
  MTWeQ: Mean Temperature of Wettest Quarter (BIO8) & 0.087 & 0.428 \\ 
  Latitude & 0.051 & 0.608 \\ 
  PS: Precipitation Seasonality (Coefficient of Variation) (BIO15) & 0.037 & 0.695 \\ 
  PDQ: Precipitation of Driest Quarter (BIO17) & 0.037 & 0.699 \\ 
  PWaQ: Precipitation of Warmest Quarter (BIO18) & 0.007 & 0.937 \\ 
   \hline
\end{tabular}
\caption{Results of the NMS ordination vector analysis.} 
\label{tab:wc_vec}
\end{table}
