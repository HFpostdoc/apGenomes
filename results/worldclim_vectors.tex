% latex table generated in R 3.4.2 by xtable 1.8-2 package
% Wed Mar 21 09:44:34 2018
\begin{table}[ht]
\centering
\begin{tabular}{rrr}
  \hline
 & r & p \\ 
  \hline
Iso: Isothermality (BIO2/BIO7) (* 100) (BIO3) & 0.344 & 0.013 \\ 
  PCQ: Precipitation of Coldest Quarter (BIO19) & 0.262 & 0.042 \\ 
  Longitude & 0.227 & 0.068 \\ 
  PWeQ: Precipitation of Wettest Quarter (BIO16) & 0.232 & 0.068 \\ 
  PA: Annual Precipitation (BIO12) & 0.226 & 0.071 \\ 
  Latitude & 0.194 & 0.110 \\ 
  PWM: Precipitation of Wettest Month (BIO13) & 0.167 & 0.154 \\ 
  PWaQ: Precipitation of Warmest Quarter (BIO18) & 0.163 & 0.160 \\ 
  Tmax: Max Temperature of Warmest Month (BIO5) & 0.146 & 0.194 \\ 
  MTDQ: Mean Temperature of Driest Quarter (BIO9) & 0.147 & 0.195 \\ 
  MTCQ: Mean Temperature of Coldest Quarter (BIO11) & 0.137 & 0.220 \\ 
  MAT: Annual Mean Temperature (BIO1) & 0.133 & 0.228 \\ 
  TS: Temperature Seasonality (standard deviation *100) (BIO4) & 0.129 & 0.243 \\ 
  Tmin: Min Temperature of Coldest Month (BIO6) & 0.124 & 0.257 \\ 
  ATR: Temperature Annual Range (BIO5-BIO6) (BIO7) & 0.106 & 0.316 \\ 
  PDM: Precipitation of Driest Month (BIO14) & 0.091 & 0.357 \\ 
  MTWaQ: Mean Temperature of Warmest Quarter (BIO10) & 0.091 & 0.366 \\ 
  MTWeQ: Mean Temperature of Wettest Quarter (BIO8) & 0.055 & 0.561 \\ 
  PDQ: Precipitation of Driest Quarter (BIO17) & 0.052 & 0.563 \\ 
  PS: Precipitation Seasonality (Coefficient of Variation) (BIO15) & 0.022 & 0.788 \\ 
  MDR: Mean Diurnal Range (Mean of monthly (max temp - min temp)) (BIO2) & 0.018 & 0.835 \\ 
   \hline
\end{tabular}
\caption{Results of the NMS ordination vector analysis.} 
\label{tab:wc_vec}
\end{table}
