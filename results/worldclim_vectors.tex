% latex table generated in R 3.5.1 by xtable 1.8-3 package
% Tue Dec 18 14:54:40 2018
\begin{table}[ht]
\centering
\begin{tabular}{rr}
  \hline
 & {\emph{r}} \\ 
  \hline
MDR: Mean Diurnal Range (Mean of monthly (max temp - min temp)) (BIO2) & 0.270 \\ 
  PCQ: Precipitation of Coldest Quarter (BIO19) & 0.252 \\ 
  PA: Annual Precipitation (BIO12) & 0.203 \\ 
  PWaQ: Precipitation of Warmest Quarter (BIO18) & 0.173 \\ 
  PWeQ: Precipitation of Wettest Quarter (BIO16) & 0.173 \\ 
  PWM: Precipitation of Wettest Month (BIO13) & 0.166 \\ 
  ATR: Temperature Annual Range (BIO5-BIO6) (BIO7) & 0.128 \\ 
  MTWeQ: Mean Temperature of Wettest Quarter (BIO8) & 0.093 \\ 
  TS: Temperature Seasonality (standard deviation *100) (BIO4) & 0.110 \\ 
  Tmin: Min Temperature of Coldest Month (BIO6) & 0.100 \\ 
  PS: Precipitation Seasonality (Coefficient of Variation) (BIO15) & 0.127 \\ 
  Iso: Isothermality (BIO2/BIO7) (* 100) (BIO3) & 0.102 \\ 
  MTCQ: Mean Temperature of Coldest Quarter (BIO11) & 0.091 \\ 
  Latitude & 0.089 \\ 
  MAT: Annual Mean Temperature (BIO1) & 0.084 \\ 
  MTWaQ: Mean Temperature of Warmest Quarter (BIO10) & 0.064 \\ 
  Longitude & 0.046 \\ 
  Tmax: Max Temperature of Warmest Month (BIO5) & 0.039 \\ 
  MTDQ: Mean Temperature of Driest Quarter (BIO9) & 0.040 \\ 
  PDQ: Precipitation of Driest Quarter (BIO17) & 0.029 \\ 
  PDM: Precipitation of Driest Month (BIO14) & 0.024 \\ 
   \hline
\end{tabular}
\caption{Results of the NMS ordination vector analysis.} 
\label{tab:wc_vec}
\end{table}
