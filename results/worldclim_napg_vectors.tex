% latex table generated in R 3.5.1 by xtable 1.8-3 package
% Sat Dec 29 22:04:23 2018
\begin{table}[ht]
\centering
\begin{tabular}{rrr}
  \hline
 & {\emph{r}} & {\emph{p-value}} \\ 
  \hline
MDR: Mean Diurnal Range (Mean of monthly (max temp - min temp)) (BIO2) & 0.356 & 0.065 \\ 
  PCQ: Precipitation of Coldest Quarter (BIO19) & 0.316 & 0.094 \\ 
  Iso: Isothermality (BIO2/BIO7) (* 100) (BIO3) & 0.276 & 0.134 \\ 
  Tmax: Max Temperature of Warmest Month (BIO5) & 0.248 & 0.154 \\ 
  PS: Precipitation Seasonality (Coefficient of Variation) (BIO15) & 0.257 & 0.160 \\ 
  PDM: Precipitation of Driest Month (BIO14) & 0.245 & 0.186 \\ 
  PDQ: Precipitation of Driest Quarter (BIO17) & 0.198 & 0.268 \\ 
  Longitude & 0.170 & 0.302 \\ 
  MTDQ: Mean Temperature of Driest Quarter (BIO9) & 0.175 & 0.302 \\ 
  MTWeQ: Mean Temperature of Wettest Quarter (BIO8) & 0.145 & 0.357 \\ 
  PWaQ: Precipitation of Warmest Quarter (BIO18) & 0.162 & 0.370 \\ 
  PA: Annual Precipitation (BIO12) & 0.145 & 0.393 \\ 
  PWeQ: Precipitation of Wettest Quarter (BIO16) & 0.129 & 0.451 \\ 
  PWM: Precipitation of Wettest Month (BIO13) & 0.129 & 0.452 \\ 
  TS: Temperature Seasonality (standard deviation *100) (BIO4) & 0.111 & 0.490 \\ 
  MTCQ: Mean Temperature of Coldest Quarter (BIO11) & 0.102 & 0.520 \\ 
  MAT: Annual Mean Temperature (BIO1) & 0.087 & 0.565 \\ 
  Tmin: Min Temperature of Coldest Month (BIO6) & 0.083 & 0.571 \\ 
  ATR: Temperature Annual Range (BIO5-BIO6) (BIO7) & 0.079 & 0.590 \\ 
  Latitude & 0.073 & 0.632 \\ 
  MTWaQ: Mean Temperature of Warmest Quarter (BIO10) & 0.061 & 0.664 \\ 
   \hline
\end{tabular}
\caption{Results of the NMS ordination vector analysis for all whole genome sequences present in NCBI prior to the current sequencing effort.} 
\label{tab:wc_napg_vec}
\end{table}
