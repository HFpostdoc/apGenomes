% latex table generated in R 3.5.2 by xtable 1.8-2 package
% Thu Feb 14 15:45:51 2019
\begin{table}[ht]
\centering
\begin{tabular}{rrr}
  \hline
 & {\emph{r}} & {\emph{p-value}} \\ 
  \hline
PCQ: Precipitation of Coldest Quarter (BIO19) & 0.355 & 0.070 \\ 
  MDR: Mean Diurnal Range (Mean of monthly (max temp - min temp)) (BIO2) & 0.303 & 0.097 \\ 
  Tmax: Max Temperature of Warmest Month (BIO5) & 0.283 & 0.115 \\ 
  Iso: Isothermality (BIO2/BIO7) (* 100) (BIO3) & 0.204 & 0.236 \\ 
  MTDQ: Mean Temperature of Driest Quarter (BIO9) & 0.199 & 0.247 \\ 
  MTWeQ: Mean Temperature of Wettest Quarter (BIO8) & 0.162 & 0.301 \\ 
  PS: Precipitation Seasonality (Coefficient of Variation) (BIO15) & 0.165 & 0.324 \\ 
  Longitude & 0.137 & 0.386 \\ 
  PDM: Precipitation of Driest Month (BIO14) & 0.150 & 0.387 \\ 
  PA: Annual Precipitation (BIO12) & 0.140 & 0.396 \\ 
  PWM: Precipitation of Wettest Month (BIO13) & 0.124 & 0.436 \\ 
  PWeQ: Precipitation of Wettest Quarter (BIO16) & 0.124 & 0.442 \\ 
  PWaQ: Precipitation of Warmest Quarter (BIO18) & 0.123 & 0.482 \\ 
  PDQ: Precipitation of Driest Quarter (BIO17) & 0.112 & 0.486 \\ 
  MTWaQ: Mean Temperature of Warmest Quarter (BIO10) & 0.100 & 0.504 \\ 
  MAT: Annual Mean Temperature (BIO1) & 0.086 & 0.567 \\ 
  MTCQ: Mean Temperature of Coldest Quarter (BIO11) & 0.085 & 0.572 \\ 
  Tmin: Min Temperature of Coldest Month (BIO6) & 0.078 & 0.582 \\ 
  ATR: Temperature Annual Range (BIO5-BIO6) (BIO7) & 0.068 & 0.623 \\ 
  TS: Temperature Seasonality (standard deviation *100) (BIO4) & 0.067 & 0.654 \\ 
  Latitude & 0.059 & 0.696 \\ 
   \hline
\end{tabular}
\caption{Results of the NMS ordination vector analysis for all whole genome sequences present in NCBI prior to the current sequencing effort.} 
\label{tab:wc_napg_vec}
\end{table}
