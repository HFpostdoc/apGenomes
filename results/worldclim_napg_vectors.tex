% latex table generated in R 3.4.2 by xtable 1.8-2 package
% Tue Mar 27 15:12:56 2018
\begin{table}[ht]
\centering
\begin{tabular}{rrr}
  \hline
 & r & p \\ 
  \hline
PCQ: Precipitation of Coldest Quarter (BIO19) & 0.393 & 0.026 \\ 
  Iso: Isothermality (BIO2/BIO7) (* 100) (BIO3) & 0.341 & 0.055 \\ 
  MTDQ: Mean Temperature of Driest Quarter (BIO9) & 0.318 & 0.063 \\ 
  Longitude & 0.252 & 0.118 \\ 
  PWM: Precipitation of Wettest Month (BIO13) & 0.246 & 0.135 \\ 
  PWeQ: Precipitation of Wettest Quarter (BIO16) & 0.229 & 0.160 \\ 
  PA: Annual Precipitation (BIO12) & 0.198 & 0.210 \\ 
  MAT: Annual Mean Temperature (BIO1) & 0.202 & 0.213 \\ 
  MTCQ: Mean Temperature of Coldest Quarter (BIO11) & 0.205 & 0.213 \\ 
  Tmin: Min Temperature of Coldest Month (BIO6) & 0.200 & 0.215 \\ 
  Tmax: Max Temperature of Warmest Month (BIO5) & 0.195 & 0.241 \\ 
  MTWaQ: Mean Temperature of Warmest Quarter (BIO10) & 0.168 & 0.286 \\ 
  TS: Temperature Seasonality (standard deviation *100) (BIO4) & 0.168 & 0.305 \\ 
  ATR: Temperature Annual Range (BIO5-BIO6) (BIO7) & 0.153 & 0.315 \\ 
  PDM: Precipitation of Driest Month (BIO14) & 0.147 & 0.364 \\ 
  Latitude & 0.123 & 0.430 \\ 
  MTWeQ: Mean Temperature of Wettest Quarter (BIO8) & 0.086 & 0.535 \\ 
  PDQ: Precipitation of Driest Quarter (BIO17) & 0.073 & 0.608 \\ 
  PS: Precipitation Seasonality (Coefficient of Variation) (BIO15) & 0.021 & 0.848 \\ 
  MDR: Mean Diurnal Range (Mean of monthly (max temp - min temp)) (BIO2) & 0.013 & 0.894 \\ 
  PWaQ: Precipitation of Warmest Quarter (BIO18) & 0.013 & 0.906 \\ 
   \hline
\end{tabular}
\caption{Results of the NMS ordination vector analysis for all whole genome sequences present in NCBI prior to the current sequencing effort.} 
\label{tab:wc_napg_vec}
\end{table}
